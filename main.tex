\documentclass[a4paper, 11pt]{article}

\usepackage[danish]{babel}
\usepackage[utf8]{inputenc}
%\usepackage{tgtermes}
%\usepackage{fouriernc}
\usepackage[T1]{fontenc}
\usepackage[margin=3cm]{geometry}

\usepackage{graphicx}
\usepackage{amssymb}
\usepackage{amsmath}
\usepackage{amsthm}
\usepackage{multicol}
\usepackage{xcolor}
\usepackage{wrapfig}
\usepackage{array} %skema

\usepackage{enumerate}
\usepackage[shortlabels]{enumitem}
\usepackage{verbatim}
\usepackage{hyperref}
\hypersetup{
    colorlinks=true,
    linkcolor=red,   
    urlcolor=red,
}
\newcommand{\N}{\mathbb{N}}
\newcommand{\Z}{\mathbb{Z}}
\newcommand{\Q}{\mathbb{Q}}
\newcommand{\R}{\mathbb{R}}

%Is defined to be equal to
\newcommand*{\defeq}{\mathrel{\vcenter{\baselineskip0.5ex \lineskiplimit0pt
                     \hbox{\scriptsize.}\hbox{\scriptsize.}}}%
                     =}
 
\title{{\large \textsc{Opgavesæt om funktioner}}}
\author{Cecilie Horshauge}
\date{\today}

\begin{document}
\maketitle
% Intro til opgaven
\noindent 

\section*{Opgave 1} 
En lineær sammenhæng er givet ved
\[y=x-9\]
\begin{enumerate}[label=\alph*)]
    \item Hvad er den afhængige variabel her?
    \item Hvad er den uafhængige variabel her?
\end{enumerate}

\section*{Opgave 2} 
En funktion er givet ved 
\[f(x)=8x-15\]
\begin{enumerate}[label=\alph*)]
    \item Bestem \(f(8)\)
    \item Bestem \(f(-1)\)
    \item Bestem \(x\), når \(f(x)=25\)
    \item Udfyld følgende tabel
\end{enumerate}
\begin{tabular}{ | m{1 cm} | m{1 cm}| m{1 cm} | m{1 cm} | m{1 cm}| m{1 cm}| m{1 cm}| m{1 cm}| m{1 cm}| }
    \hline
    \(x\) & \(-3\) & \(-2\) & \(-1\) & &  & & \(8\)\\ 
    \hline
    &\(f(x)\) & & &\(-15\)& \(9\)& \(25\)&\\ 
    \hline
  \end{tabular}\\
\section*{Opgave 3}
Indtegn punkterne fra tabellen i et koordinatsystem, er der en sammenhæng mellem punkterne?
\end{document}